\chapter{Definición del problema}
\section{Planteamiento del problema}
Pilotar una aeronave no es fácil, tanto para aeronaves tripuladas como no tripuladas es necesario tener cierto entrenamiento y demostrar la habilidad para operar de manera correcta. Debido principalmente al riesgo de que cualquier accidente es muy posiblemente catastrófico y el costo es de vidas o económico. De todas formas en las aeronaves modernas se cuentan con sistemas de piloto asistido que van desde ayudas para la  estabilización hasta completos autopilotos que ejecutan una misión desde el despegue hasta el aterrizaje \cite{autopilot-ai}, por lo que el uso apropiado reduce significativamente el riesgo de accidentes por error humano.

Para funcionar correctamente los sistemas de autopiloto necesitan conocer parámetros físicos de la aeronave que permitan predecir el movimiento en respuesta a cierto control de entrada \cite{fpvtuning}, para RPAS esta caracterización se puede realizar automáticamente en vuelo \cite{px4-autotuning}. Por ejemplo los drones o cuadricópteros primero permanecen quietos en el aire y desde ese equilibrio alteran de varias formas la velocidad de las hélices, registrando las respuestas obtenidas. Para RPAS de ala fija es similar aunque más complicado, estas aeronaves pueden alcanzar un vuelo recto y nivelado, pero deben avanzar para mantener la sustentación y cada alteración por más pequeña que sea puede traducirse en un gran desplazamiento poniendo en riesgo la aeronave.

Una posible solución para mitigar este riesgo de pilotear un RPAS de ala fija sin un controlador previamente calibrado puede ser realizar la caracterización dentro de un entorno de simulación. Si se cuenta y ejecuta algún procedimiento para crear un modelo lo suficientemente realista de la aeronave (con información obtenida en pruebas estáticas o de vuelo manual), es posible realizar una simulación y esta conectarla al software de autopiloto para que realice la calibración. Esta primera aproximación sería suficiente para comenzar las pruebas con autopiloto y permitir posterior refinamiento en vuelo.

El objetivo de esta memoria de título es implementar una arquitectura que haga de interfaz entre software de autopiloto y el simulador de vuelo X-Plane para que, contando con el modelo de la aeronave adecuado, se pueda realizar la caracterización.

\section{Objetivos}

Desarrollar una arquitectura capaz de comunicación bidireccional con X-Plane para la evaluación de diferentes controladores de vuelo automáticos en RPAS de ala fija, basándose en el uso de microcontroladores.

\subsection{Objetivos específicos}

\textbf{1.} Análisis y comparación de las capacidades de autopiloto y calibración en los software utilizados en controladores de vuelo actuales.

\textbf{2.} Selección de firmware de acuerdo al análisis realizado y las necesidades.

\textbf{3.} Preparación del protocolo de comunicación con X-Plane para ser compatible con controladores de vuelo y sus interfaces estándar.

\textbf{4.} Diseño y ejecución de pruebas de validación de la arquitectura en simulador con modelos existentes.

\textbf{5.} Redacción de informe final.

\section{Condiciones de diseño}

La fidelidad del sistema interfaz autopiloto con X-Plane en comparación a un controlador funcionando en una aeronave real debe ser tan alta como lo sea el modelo utilizado. Si se tiene un modelo idéntico al de un RPAS disponible en el laboratorio con el que realizar pruebas, los resultados de la calibración deben ser los mismos.

\subsection{Autopiloto}

Los controladores de autopiloto actuales en su mayoría están diseñados para trabajar con la familia de microcontroladores STM32 \cite{ardupilot-porting}, por lo que es este el microcontrolador en el que se desarrollara la interfaz. Esto habilita el uso de firmware como ArduPilot y PixHawk que son los más populares en RPAS.

\subsection{X-Plane}

El controlador autopiloto se debe conectar a X-Plane por la misma interfaz con la que lo haría en un RPAS, estas conexiones están estandarizadas por lo que se debe extender el protocolo de comunicación para entregar datos de la forma que el autopiloto los espera. Estos pueden ser lecturas de los sensores y los comandos de control del piloto. Los modelos de aeronave a utilizar serán los existentes proporcionados por el simulador puesto que estos son creados por los desarrolladores que aseguran que el comportamiento es comparable al de una aeronave real.

\section{Metodología de trabajo}

Primero se debe realizar un análisis de los controladores de autopiloto utilizados actualmente y de sus capacidades para trabajar con RPAS de ala fija. Las diferencias entre controladores son de capacidades macro como solo ofrecer control de estabilización y nada más, o entregar una solución completamente autónoma. A más profundidad el software puede ofrecer las mismas capacidades de control, pero implementadas con algoritmos diferentes \cite{ardupilot-vs-betaflight-vs-inav}, por lo que la selección de uno o varios controladores es importante de acuerdo a los objetivos que se quieran alcanzar y el rendimiento que pudiesen tener uno en comparación a otro. Además del rendimiento están los criterios de las interfaces con las que se comunican con la aeronave y si estas son estandarizadas, y que tan utilizado es el controlador en la práctica.

Seleccionados los controladores se debe extender el protocolo de comunicación para ser compatible con estos, el protocolo funciona con un microcontrolador Raspberry Pi Pico el cual emulara y entregara las salidas que se esperarían de un sensor con ruido además de los comandos de control del piloto. Esta nueva interfaz será construida para ser compatible con los estándares de los controladores seleccionados.

La validación de que el sistema funciona correctamente se realizara utilizando modelos existentes en el software de simulación verificando que el autopiloto los pueda controlar exitosamente. Los modelos desarrollados por profesionales ofrecen cierta garantía de que se comportan como aeronaves reales.

Finalmente, la redacción del informe abarcará el desarrollo y el uso del sistema para poder ser aplicado en el desarrollo de nuevos modelos de aeronaves no caracterizadas.